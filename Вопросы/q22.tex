\section{Гибридные схемы шифрования. Практические примеры реализации гибридных схем.}

Гибридные схемы шифрования -- протоколы, в которых используются  симметричные и асимметричные алгоритмы. Сценарий использования, как правило, один: асимметричный алгоритм используется для получения двумя сторонами общего ключа симметричного шифрования, и далее этот ключ используется для шифрования всей остальной информации. Такой ключ симметричного шифрования, как правило, называется сеансовым. 

Причины возникновения таких протоколов, в основном, в том, что:

\begin{enumerate}
	\item Ключи симметричного шифрования необходимо периодически менять, т.к., во-первых, они могут быть скомпрометированы, во-вторых, если злоумышленник накопит достаточно много единиц шифртекста, зашифрованных на одном ключе, это открывает возможности для некоторых атак.
	\item Использовать асимметричное шифрование для защиты основных данных неудобно, т.к. асимметричные алгоритмы достаточно медленны в работе (по утверждениям Шнайера, примерно в 1000 раз медленнее симметричных), а также уязвимы к атакам с выбранным открытым текстом.
\end{enumerate}

Таким образом, есть общая наиболее распространённая схема взаимодействия:

\begin{enumerate}
	\item Принимающая сторона посылает передающей свой открытый ключ.
	\item Передающая сторона зашифровывает свой сеансовый ключ симметричного шифрования открытым ключом принимающей стороны и отправляет.
	\item Принимающая сторона расшифровывает сеансовый ключ своим закрытым ключом.
	\item Таким образом, обе стороны имеют общий ключ симметричного шифрования -- сеансовый -- и все дальнейшие передаваемые данные в пределах этого сеанса связи шифруются им. 
\end{enumerate}

Из практических примеров реализации таких схем можно привести, например, протокол TLS (см. вопрос 31). Так, в версии 1.2 этого протокола в качестве одного из вариантов используется алгоритм RSA для обмена ключами и алгоритм AES в режиме GCM для шифрования основного трафика. Однако чаще в TLS (особенно в версии 1.3 используется всё же алгоритм Диффи-Хеллмана, где каждая сторона дополнительно подписывает свою часть электронной подписью).

Другими примерами являются протоколы SSH, OpenPGP и PKCS\#7.