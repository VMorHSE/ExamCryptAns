\section{Алгоритмы выработки имитовставки. Методы оценки имитозащищенности.}

Имитовставка (MAC, англ. message authentication code) -- средство обеспечения защиты от навязывания ложных данных в протоколах аутентификации сообщений с доверяющими друг другу участниками -- специальный набор символов, который добавляется к сообщению и предназначен для обеспечения его целостности и аутентификации источника данных.

Проблема. Обычное средство обеспечения целостности -- хеш функция. Передающая сторона вычисляет хеш от своего сообщения и прикрепляет к сообщению. Принимающая сторона тоже вычисляет хеш от сообщения и сверяет с прикреплённым. Если получено равенство, сообщение с высокой вероятностью неизменно. Однако при такой схеме злоумышленник тоже может посылать принимающей стороне сообщения, хешируя их тем же алгоритмом, и принимающая сторона никак не отличит подделку. 

Для решения проблемы в вычисление хеш-функции каким-либо образом вводится секретный ключ, известный только легитимным сторонам взаимодействия: отправителю и получателю. Самый простой способ -- просто зашифровать хеш сообщения каким-то симметричным алгоритмом. Есть способы сложнее, например, специальный режим выработки имитовставки в ГОСТ Р 34.13-2015 (см. вопрос 16). Также есть способ CBC-MAC, где в качестве значения имитовставки берётся последний блок сообщения зашифрованного блочным алгоритмом в режимах CBC или СFB.

У всех этих способов и у имитовставки в целом есть один большой недостаток. Они работают только в случае, если стороны обмена доверяют друг другу. Происходит это по двум причинам:

\begin{enumerate}
	\item Передающая сторона может отправить какое-то <<плохое>> сообщение, сгенерировав для него имитовставку, но потом утверждать, что получатель сам сгенерировал это сообщение, т.к. у него тоже есть ключ, и он тоже мог сгенерировать имитовставку.
	\item Принимающая сторона может сгенерировать какое-то <<плохое>> сообщение, сгенерировать имитовставку и утверждать, что получила его от передающей стороны, т.к. у передающей стороны тоже есть ключ и она могла сгенерировать имитовставку.
\end{enumerate}

В каждом из этих случаев в случае использования имитовставки отсутствует возможность проверить, кто из сторон говорит правду. Эта проблема решается с помощью электронной подписи. 