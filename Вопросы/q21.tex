\section{Асимметричные криптографические схемы.}

Асимметричные криптографические схемы -- протоколы криптографии с открытым ключом.

Основная суть состоит в следующем: каждая сторона информационного обмена имеет пару ключей: открытый и закрытый. При этом из закрытого можно легко вычислить открытый, а вот наоборот -- вычислительно сложно. 

Открытый ключ публикуется для всех, в т.ч. его может знать злоумышленник. Это нормально, потому что открытый ключ может использоваться только для зашифрования сообщений. Расшифровать такое сообщение можно только с помощью закрытого ключа, который каждая из сторон держит в секрете. 

Преимущество в том, что за всё время информационного обмена ни разу не возникает необходимости куда-либо передавать закрытый ключ. Сторона генерирует его один раз и всё время держит у себя. 

Чтобы передать сообщение, передающая сторона сначала запрашивает открытый ключ принимающей стороны и потом зашифровывает своё сообщение на этом ключе. Таким образом, только принимающая сторона сможет расшифровать это сообщение.

Такая система уязвима к некоторым атакам, т.к. открытый ключ общеизвестен и злоумышленник может нашифровать себе сколько угодно сообщений и потом просто сопоставлять передаваемый шифртекст со своим набором шифртекстов. 

Другая проблема состоит в том, что передающей стороне зачастую необходимо убедиться, что тот ключ, который пришёл её от принимающей стороны, действительно является открытым ключом принимающей стороны, а не <<человека посередине>>, вторгшегося в канал связи. Для этого используется инфраструктура открытых ключей (см. вопрос 23).

Ряд алгоритмов с открытым ключом пригодны также для создания цифровой подписи (см. вопрос 23).

Если переходить к конкретным алгоритмам, из распространённых можно назвать RSA, ElGamal, Rabin и, конечно же, ГОСТ Р 34.10-2012 (который используется только для создания подписи). 

Сложность расшифрования шифртекста без знания закрытого ключа, как правило, базируется на сложности некоторой математической проблемы. При этом, зная решение этой проблемы (закрытый ключ), расшифровать информацию достаточно просто. Так, шифр RSA базируется на сложности задачи факторизации больших чисел -- разложения этих чисел на простые множители. ElGamal, в свою очередь, основана на сложности вычисления дискретных логарифмов в конечном поле. Rabin -- на сложности поиска квадратных корней в кольце остатков по модулю составного числа. ГОСТ -- на сложности вычисления дискретного логарифма в группе точек эллиптической кривой. 

Многие существующие схемы криптографии с открытым ключом сейчас имеют аналоги с использованием эллиптических кривых.