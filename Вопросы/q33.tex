\section{Протокол SESPAKE выработки общего ключа на основе пароля, область его применения, принципы обоснования сложности перебора паролей.}

SESPAKE по сути является большим расширением над протоколом Диффи-Хеллмана, позволяющим защитить выработку общего ключа с помощью слабого секрета -- пароля.

Сам протокол описан в Рекомендациях по стандартизации Р 50.1.115-2016 <<Протокол выработки общего ключа с аутентификацией на основе пароля>>, включает 27 шагов и большое количество различных вспомогательных значений, так что здесь будут описаны только основные принципы, лежащие в его основе. 

В ходе протокола Диффи-Хеллмана две стороны открыто согласуют два числа $p$ и $g$, после чего отправляют друг другу значения $g^a\ mod\ p$ и $g^b\ mod\ p$, где $a$ -- сгенерированное одной стороной секретное случайное число, $b$ -- аналогичное число другой стороны. Потом каждая сторона вычисляет $(g^a)^b\ mod\ p = (g^b)^a\ mod\ p$ и принимает полученное значение за секретный ключ. Известная проблема данного протокола состоит в том, что он уязвим к атаке <<человек посередине>>, когда активный злоумышленник вторгается в линию связи и может отменять пересылку некоторых сообщений и заменять их на другие. 

Чтобы избежать описанной проблемы, можно применить аутентификацию на основе пароля. Для этого необходимо, чтобы каждый из участников заранее знал некоторый пароль $S$, одинаковый для обоих участников. Тогда протокол Диффи-Хеллмана можно дополнить пересылкой сообщений вида:

$$ A \rightarrow B: H(S || K_A || 1) $$
$$ B \rightarrow A: H(S || K_B || 2) $$

Где $K_A$ и $K_B$ -- ключи, полученные сторонами в результате выполнения протокола Диффи-Хеллмана, и, если злоумышленник не вторгся в процесс работы, $K_A = K_B$. 

Такой подход позволяет сторонам убедиться, что они получили один и тот же ключ, причём получили его именно они и никто больше. 

Однако, если учесть, что $S$ -- пароль, то есть некоторая легко перебираемая машинно величина, можно увидеть, что становится возможной следующая атака. Злоумышленник может подменить собой сторону $B$ и участвовать в протоколе до тех пор, пока не получит $H(S || K_A || 1)$ от $A$, после чего прервать связь. Т.к. злоумышленник сам участвовал в выработке ключа, он знает $K_A$, а значит, $H(S || K_A || 1)$ даёт ему критерий подбора пароля. Т.е. теперь злоумышленник может без какой-либо связи с легитимными сторонами взаимодействия подставлять в хеш-функцию все подбираемые значения паролей, дополнять их ключом $K_A$ и значением $1$ и проверять, верный ли получился хеш. Когда станет верный, злоумышленник подобрал пароль. Такая атака называется offline dictionary attack. 

Чтобы такого не происходило, нужно включить пароль в сам процесс генерации ключа. Именно на этой идее основаны PAKE-протоколы (PAKE -- Password Authenticated Key Exchange).

Для реализации простейшего из таких протоколов стороны должны согласовать и открыто опубликовать ещё одно число -- $h$. При этом важно, чтобы дискретные логарифмы $h$ по основанию $g$ и наоборот были неизвестны, т.е. $h$ и $g$ должны быть сгенерированы случайно. Протокол выглядит следующим образом:

\begin{enumerate}
	\item $A \rightarrow B: M = g^ah^S$
	\item $B \rightarrow A: N = g^bh^S$
	\item $A: \text{вычисляет}\ K_A = (N/h^S)^a$
	\item $B: \text{вычисляет}\ K_B = (N/h^S)^b$
	\item $A \rightarrow B: H(K_A || 1)$
	\item $B \rightarrow A: H(K_B || 2)$
\end{enumerate}

При таком подходе злоумышленник уже не может подменить собой Боба, т.к. не знает пароль,  а чтобы получить критерий для перебора пароля, ему нужно будет как-то вычислить $K_A$, что является вычислительно сложной задачей, лежащей в основе протокола Диффи-Хеллмана.

Однако протокол можно сделать ещё безопаснее. Можно сделать так, чтобы, даже зная $K_A$ и $K_B$ злоумышленник не смог получить доступ к паролю. Для этого достаточно ввести хеширование на шагах 3 и 4, т.е., например, шаг 3 станет выглядеть следующим образом:

$$A: \text{вычисляет}\ K_A = H((N/h^S)^a)$$

На таких принципах построен протокол SESPAKE. Однако в нём используется версия Диффи-Хеллмана на эллиптических кривых вместо классической, а также много дополнительных значений и шагов для ещё большего повышения безопасности. 

Применяется этот протокол, в основном, при взаимодействии с функциональными ключевыми носителями (ФКН). ФКН -- специальные активные устройства для хранения ключа, которые никогда не передают сам ключ по каналу связи, но при этом могут сами выполнить необходимые действия с использованием ключа, например, подписать документ. Выглядеть это может, например, как флешка или смарт-карта, в которую встроены собственные процессор, память и т.д.

Без протокола SESPAKE пользователю приходилось передавать пароль на это устройство, после чего оно могло выполнить необходимую операцию, но злоумышленник в канале передачи мог получить доступ к паролю. С протоколом SESPAKE устройство соединяется со специальным ПО для ввода пароля, которое отрабатывает протокол и, соответственно, препятствует передаче пароля или какой-либо информации о нём по каналу связи.

