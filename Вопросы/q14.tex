\section{Генераторы псевдослучайных последовательностей и их свойства.}

Лекуер предложил следующее обобщённое описание детерминированного ГПСЧ.

ГПСЧ описывается следующим набором параметров: $(S, \mu, f, U, g)$.

\begin{itemize}
	\item $S$ -- множество состояний ГПСЧ;
	
	\item $\mu$ -- распределение вероятностей выбора начального состояния $s_0$;
	
	\item $f$ -- функция перехода $f: S \rightarrow S, s_i = f(s_i - 1)$;
	
	\item $U$ -- множество выходных значений, как правило, $U = \{0, 1\}$;
	
	\item $g: S \rightarrow U$.
\end{itemize}

Периодом ГПСЧ называют число переходов, после которого состояния и, следовательно, генерируемые значения начинают повторяться.

Примеры ГПСЧ:

\subsection{Линейный конгруэнтный метод}

Линейный конгруэнтный метод -- один из самых простых, но популярных. Каждое новое случайное число $X_i = (aX_i - 1 + c)\ mod\ m$, где $a$, $c$ и $m$ -- некоторые константы. Период не может быть больше $m$, но можно сделать его равным $m$ при определённых условиях.  Не используется в криптографии, т.к. значения предсказуемы. На основе этого генератора разрабатывались многие другие, например, полиномиальные конгруэнтные методы, но для них всех также было показано, что значения можно предсказать. Также стоит отметить, что биты генерируемых чисел зачастую не равновероятны, поэтому часто берут только некоторые битовые части получившегося числа. Однако их использование для моделирования вполне оправдано.

\subsection{Запаздывающие генераторы Фибоначчи (аддитивные генераторы)}

Это целое семейство генераторов, суть которых в том, что каждое следующее число зависит от нескольких уже сгенерированных сколько-то шагов назад. В общем виде аддитивные генераторы работают по следующей формуле: $X_i = (X_{i - a} + X_{i - b} + X_{i - c} + ... + X_{i - m})\ mod\ 2^n$. Соответственно, в качестве начального состояния необходимо задать массив из m значений. Если хорошо подобрать коэффициенты $a$, $b$, $c$ и т.д., можно сделать так, чтобы период генератора был не меньше $2n - 1$. Например, распространённый фибоначчиев генератор работает по такой формуле:

$$ X_k =
\begin{cases}
	X_{k - a} - X_{k - b}, & \text{если}\ X_{k - a} \geq X_{k - b}, \\
	X_{k - a} - X_{k - b} + 1, & \text{если}\ X_{k - a} < X_{k - b},
\end{cases}
$$

В данном случае $X_i$ -- вещественные числа. Для работы генератору нужно держать в памяти $max(a, b)$ чисел, изначально они могут быть сгенерированы конгруэнтным методом. Период такого генератора может быть оценен как $T = (2^{max(a, b)} - 1) \cdot 2^l$, где $l$ -- число битов в мантиссе вещественного числа.

Фибоначчиевы генераторы создают последовательности, обладающие хорошими статистическими свойствами, причём для всех бит.

\subsection{Регистры сдвига с линейной обратной связью (LFSR)}

LFSR -- сдвиговый регистр битовых слов, у которого значение входного (вдвигаемого) бита равно линейной булевой функции от значений остальных битов регистра до сдвига. Используется как ГПСЧ для генерации одиночных бит. 

Булева функция, как правило, строится следующим образом. Каждый из $L$ бит регистра умножается на некоторый коэффициент $c_i \in \{0, 1\}, i = \overline{1, L}$, причём $c_L = 1$. Дальше все результаты умножения складываются xor-ом. Таким образом, коэффициенты $c_i$ формируют характеристический многочлен степени $L$ над полем $GF(2)$ для данного регистра. Чтобы период генерации случайных чисел был максимальным ($2^L - 1$), нужно, чтобы многочлен был примитивным. 

Основные недостатки: медленно работает, многочлен можно определить, получив $2L$ бит с генератора.

\subsection{Регистры сдвига с обобщённой обратной связью (GFSR)}

GFSR -- продолжение развития LFSR. Чтобы построить GFSR нужно:

\begin{enumerate}
	\item Взять LFSR с примитивным характеристическим трёхчленом $x^p + x^{p - q} + 1$, где $p$ и $q$ -- некоторые натуральные числа, $q < p$.
	\item Сгенерировать неповторяющуюся последовательность максимальной длины ($2^p - 1$), произвольно задав начальное состояние. 
	\item Составить таблицу из $p$ строк, в которой каждая строка -- сгенерированная на предыдущем шаге последовательность, циклически сдвинутая на некоторое фиксированное кол-во бит вправо. 
	\item Столбцы этой таблицы и есть значения нового генератора, их тоже $2^p - 1$ штук, но каждое из них уже представляет не бит, а $p$-битное слово.
\end{enumerate}

\subsection{Вихрь Мерсенна}

Вихрь Мерсенна -- очень популярная вариация GFSR. Используется во многих ЯП, например, в C++. Есть несколько вариаций, но самая распространённая -- MT19937 ($2^{19937} - 1$ -- период генератора).

Строится он похоже на GFSR:

\begin{enumerate}
	\item Выстраивается таблица из 624 строк $W_i$, каждая по 32 бита.
	\item Генерируется промежуточная строка $tmp$, равная первому (старшему) биту первой строки таблицы, конкатенированному с младшими 31 битами второй строки таблицы.
	\item Полученная строка $tmp$ объединяется xor-ом с 227-ой строкой таблицы и таким образом получается случайное 32-битное число. 
	\item Таблица сдвигается вверх удалением первой строки и добавлением полученного случайного числа в качестве последней строки.
\end{enumerate}

Генератор получил такую популярность благодаря, во-первых, огромному периоду ($2^{19937} - 1$), во-вторых, прекрасным статистическим свойствам.
