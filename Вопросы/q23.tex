\section{Электронная подпись, инфраструктура открытых ключей. Удостоверяющие центры. Методы обеспечения подлинности физических лиц.}

В конце ответа на вопрос 17 была поставлена проблема, связанная с тем, что, используя имитовставку, стороны обмена, не доверяющие друг другу, могут приходить к неразрешимым спорам в результате злонамеренных действий одной из сторон. Ключевым моментом в данной проблеме было то, что, т.к. обе стороны информационного обмена владеют общим секретным ключом, они могут сгенерировать имтовставку для любого сообщения, при этом сообщение никак не привязывается к конкретному выработавшему её пользователю. Таким образом, возникает необходимость в алгоритме, который позволил бы производить похожие действия, но так, чтобы можно было уникально идентифицировать пользователя, создавшего некоторую последовательность для контроля целостности, т.е. необходимо обеспечить не только целостность и аутентичность информации, но и неотказуемость стороны информационного обмена от авторства. 

Для этого используется криптография с открытым ключом. У каждого из пользователей есть ключевая пара -- открытый и закрытый ключи, при этом открытый известен всем, закрытый -- только самому пользователю. Примечательно, что в большинстве алгоритмов с открытым ключом можно использовать две вариации: открытый ключ для зашифрования и закрытый для расшифрования (классический вариант для обеспечения конфиденциальности), и наоборот -- закрытый для зашифрования и открытый для расшифрования (то, что нам нужно сейчас). 

Таким образом, чтобы обеспечить все необходимые свойства, можно использовать следующий алгоритм. Отправитель может зашифровать отправляемое сообщение своим закрытым ключом и передавать полученный шифртекст вместе с самим сообщением. Тогда получатель, зная открытый ключ отправителя, может его расшифровать и сравнить с сообщением. Соответственно, если получателю удалось расшифровать сообщение открытым ключом отправителя, и сравнение прошло успешно, получатель может быть уверен во-первых, в том, что информация не была подменена, во-вторых, в том, что информацию отправил именно тот отправитель, открытый ключ которого использовался для расшифрования. Более того, получатель теперь не сможет сам сформировать сообщение и выдавать его за сообщение отправителя, т.к. не знает закрытого ключа. Отправитель, в свою очередь, не сможет сформировать сообщение, а потом отказаться от него, т.к. он зашифровал его своим закрытым ключом и каждый, кто знает его открытый ключ, может проверить, что это сделал именно он.

Описанный алгоритм работает достаточно медленно, т.к. сообщения могут быть длинными и приходится зашифровывать всё. Для этого зашифровывается не само сообщение, а его хеш. Полученный элемент данных фиксированной длины называется электронной подписью данного сообщения. Тогда получение такого элемента данных называется формированием подписи (подписанием), а расшифрование и сравнение хеша с хешем сообщения -- проверкой подписи. 

Однако остаётся ещё одна проблема. Чтобы все знали открытый ключ некоторого пользователя и могли проверять его подпись, нужно, чтобы этот ключ хранился где-то в открытом доступе. Однако при доступе к этому хранилищу в сеанс связи может вторгнуться злоумышленник (человек посередине) и подменить ключ некоторого пользователя своим. Тогда злоумышленник сможет подписывать произвольную информацию от имени легитимного пользователя. 

Чтобы этого избежать, организуется инфраструктура открытого ключа (Public Key Infrastructure -- PKI) -- система, с помощью которой можно определить, кому принадлежит тот или иной открытый ключ. Ключевыми сущностями в такой системе являются сертификаты и центры сертификации. 

Цифровой сертификат -- способ сопоставления открытого ключа физическому лицу или уполномоченному агенту, лежащий в основе инфраструктуры открытого ключа. 

Центр сертификации (центр сертификатов, удостоверяющий центр) -- это центральный орган, служащий посредником между пользователями и удостоверяющий аутентичность их открытых ключей. 

Центр сертификации представляет собой организацию, которой по той или иной причине доверяют все стороны информационного обмена. Центр сертификации имеет свою ключевую пару для электронной подписи. Он выдаёт сторонам информационного обмена цифровые сертификаты -- электронные документы, содержащие записи вида $$\{\text{Уникальный ID пользователя}, \text{Открытый ключ пользователя}\}$$, подписанные подписью центра сертификации (полное содержимое сертификата регламентируется стандартом X.509). 

Таким образом, чтобы удостовериться в подлинности ключа стороны А, сторона Б может обратиться к центру сертификации и запросить у него сертификат ключа стороны А. Также каждая из сторон может попросить центр сертификации выдать ей сертификат на какой-то новый ключ.

Проблема, которая остаётся: физически невозможно организовать один центр сертификации на весь мир, значит, нужно несколько. Для этого центры сертификации (ЦС) организуются в иерархическую структуру (дерево), где есть корневой ЦС, котрому доверяют все (например, в России это Минцифры), и который выдаёт сертификаты другим подчинённым ЦС. Те, в свою очередь, могут выдавать сертификаты другим ЦС или конкретным пользователям и т.д. 

При такой иерархической структуре, чтобы убедиться в подлинности ключа стороны А, сторона Б должна проверить сертификат, выданный стороне А его ЦС, потом сертификат этого ЦС от другого ЦС и т.д., пока не дойдёт до ЦС, которому доверяет сторона Б. Чтобы сократить этот путь по дереву, существует перекрёстная сертификация, когда узлы дерева выдают сертификаты друг другу.