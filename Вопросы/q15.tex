\section{Блочные и поточные шифры.}

Блочные шифры преобразуют открытый текст в шифртекст блоками по несколько байт (в зависимости от реализации шифра), поточные, как правило, преобразуют по одному байту.

Общее требование к блочным шифрам, сформированное Шенноном -- принцип “перемешивания” -- гласит, что незначительные изменения открытого текста должны приводить к значительным изменениям шифртекста.

Для выполнения этого требования блочные шифры строятся следующим образом. При зашифровании над каждым блоком открытого текста итерационно повторяются два типа преобразований: криптографически сложные преобразования частей блока и простые перестановки частей в пределах блока.

Примеры блочных шифров: AES, ГОСТ Р 34.12-2015.