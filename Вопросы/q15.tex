\section{Блочные и поточные шифры.}

Блочные шифры преобразуют открытый текст в шифртекст блоками по несколько байт (в зависимости от реализации шифра), поточные, как правило, преобразуют по одному байту или (если верить Шнайеру) биту. Ввиду такой неоднозначности с “маленькой порцией сообщения”, преобразуемой поточным шифром за один раз, Шнайер приводит следующий способ различия блочных и поточных шифров: блочный шифр преобразует данные большими блоками с помощью фиксированного преобразования, а поточный применяет изменяющееся во времени преобразование к отдельным символам открытого текста. 

Общее требование к блочным шифрам, сформированное Шенноном -- принцип “перемешивания” -- гласит, что незначительные изменения открытого текста должны приводить к значительным изменениям шифртекста.

Для выполнения этого требования блочные шифры строятся следующим образом. При зашифровании над каждым блоком открытого текста итерационно повторяются два типа преобразований: криптографически сложные преобразования частей блока и простые перестановки частей в пределах блока.

Примеры блочных шифров: AES, ГОСТ Р 34.12-2015.

Поточные шифры реализуются по следующему общему принципу. Есть некоторый генератор потока ключей, который по определённому правилу генерирует биты некоторой гаммы, зависящей от ключа симметричного шифрования. Для зашифрования генерируемая гамма, бит за битом, накладывается на соответствующие биты открытого текста с помощью xor. Для расшифрования та же самая гамма должна быть наложена на биты шифр текста. 

Очевидной проблемой в данном подходе является синхронизация. Генераторы ключевых последовательностей должны работать идентично на передающей и приёмной сторонах. К такой синхронизации существуют два общих подхода:

\begin{enumerate}
	\item Самосинхронизация по шифртексту. В таком случае каждый следующий бит гаммы является функцией от основного ключа и $n$ предыдущих бит шифртекста. Этот подход в некоторой степени устойчив как к потере, так и к искажению отдельного бита. В обоих случаях будут искажены только $n$ бит расшифрованного текста.
	
	\item  Синхронные потоковые шифры. Здесь поток ключей генерируется независимо от сообщения. Здесь обе стороны всегда должны иметь полностью синхронизированные состояния своих генераторов. Для этого, как правило, перед началом работы задаётся какое-то начальное состояние. Т.к. смена состояния зависит только от номера текущего зашифровываемого или расшифровываемого бита, такой подход неустойчив к потере бита. Одна потеря приведёт к неверному расшифрованию всего последующего сообщения. Чтобы избежать этого, используются, например, специальные “маркеры” -- некоторые данные вставляемые в сообщение с определённым периодом и показывающие, в каком состоянии должен находиться генератор к моменту расшифрования данного бита. Однако такой подход гораздо более устойчив к искажению бита. Искажение одного бита повлияет только на соответствующий бит полученного расшифрованного текста и ни на какие другие. Если учесть, что потеря бита является гораздо более редкой ситуацией, чем искажение, этот подход не выглядит так уж плохо. 
\end{enumerate}

Стоит отметить, что блочные шифры при использовании в некоторых режимах (см. вопрос 16) по сути представляют собой поточные шифры с самосинхронизацией, где в качестве размера символа берётся размер блока.

Поточные шифры чаще всего реализуются с помощью ГПСЧ (см. вопрос 14). Например, чтобы реализовать генератор ключевой последовательности с помощью LFSR, достаточно взять несколько LFSR с разными характеристическими многочленами и длиной, а начальное состояние задать с помощью ключа. Дальше для генерации нового ключевого бита используется сдвиг всех регистров, а ключевой бит определяется как функция некоторых бит из этих регистров. При этом LFSR могут быть по-разному связаны между собой и влиять друг на друга. Например, Шнайер приводит несколько генераторов вида Stop-and-Go, в которых выход одного LFSR определяет, будет ли осуществляться сдвиг одного или нескольких других LFSR или нет. 