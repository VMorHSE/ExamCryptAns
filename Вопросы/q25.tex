\section{Свойства безопасности криптографических протоколов.} 

Здесь будет прямой копипаст из работы А. Ю. Нестеренко, А. М. Семенов -- Методика оценки безопасности криптографических протоколов.

Выделяют следующие свойства безопасности криптографических протоколов:

\begin{enumerate}
	\item Свойство аутентификации субъекта (участника протокола) другим субъектом (участником протокола) заключается в подтверждении одним субъектом подлинности другого субъекта, а также в получении гарантии того, что субъект, подлинность которого подтверждается, действительно принимает участиев выполнении текущей сессии протокола.Свойство аутентификации субъекта может быть как односторонним, так и взаимным. В последнем случае свойство должно выполняться для всех участвующих во взаимодействии субъектов.

	\item Свойство аутентификации сообщения заключается в подтверждении подлинности источника сообщения и целостности передаваемого сообщения. Подлинность источника сообщения означает, что протокол должен обеспечивать гарантии того, что полученное сообщение или его часть были созданы участником взаимодействия в ходе выполнения текущей сессии протоколав некоторый момент времени, предшествующий получению сообщения. Фактически в рамках данного свойства сообщение однозначно связывается со своими сточником (субъектом, отправившим сообщение), а выполнение свойства гарантирует, что сообщение не было искажено, в частности подделано нарушителем, при передаче по каналам связи.
	
	\item Свойство целостности сообщений заключается в том, что получатель сообщения обладает возможностью проверить, что полученные им данные (или их часть) не были модифицированы, уничтожены и являются теми же самыми данными, что послал отправитель. 
	
	\item Свойство защиты от повторов заключается в том, что один раз корректно принятое участником протокола сообщение не должны быть принято повторно. В зависимости от протокола данное свойство может быть сформулировано в виде одного из следующих требований:
	
	\begin{itemize}
		\item должна быть обеспечена гарантия того, что сообщение выработано в рамках текущей сессии протокола;
		\item должна быть обеспечена гарантия того, что сообщение выработано в рамках заданного интервала времени;
		\item сообщение не было принято ранее.
	\end{itemize}
	
	В отечественной литературе данное свойство часто называют свойством невозможности навязывания ложных сообщений, подразумевая под этим защиту как от повторного принятия истинных сообщений, так и от подделанных нарушителем сообщений (свойство C 2).
	
	\item Свойство неявной аутентификации получателя заключается в том, что протокол должен обладать средствами, гарантирующими, что отправленное сообщение может быть прочитано только теми участниками, для которых оно предназначено. Только законные авторизованные участники должны иметь доступ к данной информации, многоадресным сообщениям или групповому взаимодействию.
	
	\item Свойство групповой аутентификации заключается в том, что законные авторизованные члены заранее определённой группы пользователей могут аутентифицировать источник и содержание информации или группового сообщения. Сюда также входят протоколы, в которых участники группового взаимодействия не доверяют друг другу.
	
	\item Свойство аутентификации субъекта (участника протокола) доверенной третьей стороной. В протоколах, явно реализующих взаимодействие участников с доверенной третьей стороной, данное свойство эквивалентно первому из перечисленных свойств. В случае использования инфраструктуры открытых ключей данное свойство может выполняться косвенно, путём заверения открытых ключей участников взаимодействия электронной подписью удостоверяющего (доверенного) центра; при этом привязка аутентификации субъекта к какой-либо сессии протокола не может быть обеспечена.
	
	\item Свойство конфиденциальности ключа предполагает, что в ходе информационного взаимодействия значение ключа не может стать известным нарушителю, а также легитимным пользователям информационной системы, для которых данный ключ не предназначен. Данное свойство может применяться как к исходной ключевой информации, так и к производным сессионным ключам.
	
	\item Свойство аутентификации ключа предполагает, что один из участников взаимодействия получает подтверждение того, что никакой другой участник, кроме заранее определённого второго участника и, возможно, доверенного центра, не может обладать секретным ключом, выработанным в ходе выполнения протокола.
	
	\item Свойство подтверждения ключа заключается в том, что один из участников взаимодействия получает подтверждение того, что второй участник (или группа участников) действительно обладает заданным секретным ключом и/или имеет доступ к информации, необходимой для выработки заданного секретного ключа.
	
	\item Свойство стойкости при компрометации производных ключей состоит в том, что компрометация производных ключей, т. е. ключей, используемых непосредственно для шифрования и имитозащиты передаваемой информации, не приводит к нарушению других свойств безопасности как в рамках текущей, так и в других сессиях протокола, в частности к компрометации производных ключей, выработанных ранее или планируемых к выработке в дальнейшем. В литературе данное свойство часто называют защитой от <<чтения вперед/назад>> или используют термин <<perfect forward secrecy>>.
	
	\item Свойство стойкости при компрометации ключа аутентификации состоит в том, что компрометация долговременного ключа аутентификации не приводит к нарушению конфиденциальности информации, переданной до момента компрометации ключа, а в случае пассивного нарушителя -- и к нарушению конфиденциальности информации, передаваемой после завершения текущей сессии протокола. В литературе данное свойство иногда называют защитой от <<чтения назад>>.
	
	\item Свойство формирования новых ключей заключается в том, что протокол, обладающий данным свойством, позволяет формировать уникальные сессионные и/или производные ключи для каждой сессии протокола.
	
	\item Свойство защиты от навязывания ключевых значений гарантирует, что ни один из участников протокола не может навязать значение общего секретного, сессионного или производного ключа по своему выбору другому участнику протокола.
	
	\item Свойство защиты от навязывания параметров безопасности гарантирует, что используемые в ходе выполнения протокола или согласуемые на этапе установления соединения параметры безопасности не могут быть навязаны нарушителем. В качестве параметров безопасности могут выступать наборы используемых криптографических преобразований, численные параметры алгоритмов и алгебраических структур, в которых выполняется протокол, случайные значения, вырабатываемые в ходе выполнения протокола и т. п.
	
	\item Свойство конфиденциальности заключается в том, что данные, передаваемые в ходе информационного взаимодействия, не могут стать известными нарушителю и/или легитимным участникам, для которых они не предназначены. Легко видеть, что нарушение свойства конфиденциальности ключевой информации (C 8) приводит к нарушению конфиденциальности передаваемых данных.
	
	\item Свойство инвариантности отправителя заключается в том, что на протяжении выполнения всего протокола получатель сообщений сохраняет уверенность в том, что источник сообщения остался тем же, что и источник, с которым было начато взаимодействие (сессия протокола).
	
	\item Свойство анонимности субъекта (участника протокола) состоит в том, что нарушитель, осуществляющий перехват сообщений, не должен иметь возможность связать сообщения одного из участников с самим участником или его идентификатором.
	
	\item Свойство анонимности субъекта для других участников заключается в том, что каждый участник взаимодействия не должен иметь возможность узнать реальную личность других участников, а должен взаимодействовать с их псевдонимом или случайным идентификатором.
	
	\item Свойство защищённости от атак <<отказ в обслуживании>> подразумевает, что реализующее протокол средство защиты информации обеспечивает алгоритмические, технические и организационно-штатные меры защиты от указанного типа атак. Теоретическое исследование протокола может лишь проверить наличие алгоритмических мер, обеспечивающих защиту от данного класса атак, а также наличие эксплуатационной документации, содержащей описание технических и организационно-штатных мер защиты. В рамках предлагаемой методики представляется возможным получить лишь тривиальное численное значение показателя эффективности для данного свойства.
	
	\item Свойство защищённости от утечек по скрытым (логическим) каналам подразумевает, что протокол содержит реализацию алгоритмических мер защиты от атак, реализуемых нарушителем путём применения непредусмотренных коммуникационных каналов передачи информации. Отметим, что современные транспортные протоколы, такие, как ESP, IPSec или ADTP FIOT, содержат ряд мер, предназначенных для обеспечения данного свойства. Классификация угроз безопасности, реализуемых с использованием скрытых каналов, модель нарушителя и перечень мер защиты информационной системы от атак с использованием скрытых каналов должны разрабатываться на основе стандартов.
	
	\item Свойство защищённости от KCI-атак. Под KCI-атакой (атакой имперсонификации при компрометации долговременного секретного ключа) понимается атака, при выполнении которой нарушитель, получивший доступ к долговременному секретному ключу участника, может выдать себя перед ним за любого другого участника в рамках текущей или будущей сессии выполнения протокола. Свойство считается выполненным, если KCI-атака невыполнима.
	
	\item Свойство защищённости от UKS-атак. Под UKS-атакой понимается последовательность действий нарушителя, в результате которой законные авторизованные участники в процессе информационного взаимодействия вырабатывают общий ключ, но один из участников считает, что он выработал общий ключ с третьим участником (навязанным нарушителем в ходе выполнения протокола). При этом компрометации общего ключа как таковой не происходит, но нарушается требование аутентификации участников. Свойство считается выполненным, если подобная ситуация невозможна.
	
	\item Свойство невозможности отказа от совершённых действий представляет собой возможность проследить за всеми действиями участника взаимодействия. Согласно Р 1323565.1.012-2017, разд. 6.1.14, данное свойство должно обеспечиваться средством криптографической защиты информации, реализующим криптографический протокол.
	
	\item Свойство доказательства происхождения заключается в неоспоримом доказательстве отправки сообщения.
	
	\item Свойство доказательства доставки заключается в неоспоримом доказательстве получения сообщения.
	
	\item Свойство целостности множества состояний (криптографическое связывание состояний) заключается в том, что все участники информационного взаимодействия после выполнения протокола (или его части) в рамках одного сеанса связи имеют одинаковое представление обо всех участниках этого сеанса и выполняемых ими ролях, а также о состоянии выполнения протокола.
\end{enumerate}

При этом среди них есть базовые свойства (выполнение которых зависит от сложности решения математических задач, используемых в криптографических преобразованиях)  и есть те, которые зависят от базовых. Зависимость показана в таблице.

\begin{center}
	
	\newcolumntype{L}[1]{>{\raggedright\arraybackslash}p{#1}}
	
	\centering
	\begin{tabular}{ |L{0.7 \textwidth}|L{0.3 \textwidth}| }
		\hline
		Свойство & Зависимость \\
		\hline
		C 1 -- аутентификации участника протокола другим участником & Базовое \\
		\hline
		C 2 -- аутентификации сообщения & C 1, C 3 \\
		\hline
		C 3 -- целостности сообщений & Базовое \\
		\hline
		C 4 -- защиты от повторов & Базовое \\
		\hline
		C 5 -- неявной аутентификации получателя & C 1, C 9 \\
		\hline
		C 6 -- групповой аутентификации & C 1, C 9 \\
		\hline
		C 7 -- аутентификации субъекта доверенной третьей стороной & C 1 \\
		\hline
		C 8 -- конфиденциальности ключа & Базовое \\
		\hline
		C 9 -- аутентификации ключа & C 1, C 2, C 10, C 15 \\
		\hline
		C 10 -- подтверждения ключа & Базовое \\
		\hline
		C 11 -- стойкости при компрометации производных ключей & C 13 \\
		\hline
		C 12 -- стойкости при компрометации ключа аутентификации & C 13 \\
		\hline
		C 13 -- формирования новых ключей & C 15 \\
		\hline
		C 14 -- защиты от навязывания ключевых значений & C 1, C 3 \\
		\hline
		C 15 -- защиты от навязывания параметров безопасности & C 1, C 2, C 3 \\
		\hline
		C 16 -- конфиденциальности & C 3, C 9, C 10 \\
		\hline
		C 17 -- инвариантности отправителя & C 1, C 9 \\
		\hline
		C 18 -- анонимности субъекта & Базовое \\
		\hline
		C 19 -- анонимности субъекта для других участников & Базовое \\
		\hline
		C 20 -- защищённости от атак <<отказ в обслуживании>> & Базовое \\
		\hline
		C 21 -- защищённости от утечек по скрытым (логическим) каналам & Базовое \\
		\hline
		C 22 -- защищённости от KCI-атак & C 1, C 9, C 10, C 12, C 13, C 14 \\
		\hline
		C 23 -- защищённости от UKS-атак & C 1, C 9, C 10, C 15, C 27 \\
		\hline
		C 24 -- невозможности отказа от совершенных действий & C 25, C 26, C 27 \\
		\hline
		C 25 -- доказательства происхождения & C 1, C 2, C 9 \\
		\hline
		C 26 -- доказательства доставки & C 9, C 10 \\
		\hline
		C 27 -- целостности множества состояний & C 1, C 17, C 22, C 23 \\
		\hline
	\end{tabular}
\end{center}