\section{Протоколы семейства TLS, область их применения, методы оценки безопасности.}

TLS (англ. transport layer security — Протокол защиты транспортного уровня) -- криптографический протокол, обеспечивающий защищённую передачу данных между узлами в сети Интернет.

Использует асимметричное шифрование для аутентификации, симметричное шифрование для конфиденциальности и коды аутентичности сообщений для сохранения целостности сообщений.

Область применения: прежде всего, HTTPS, т.е. практически все веб-браузеры и сайты соответственно. Однако также используется в электронной почте, чатах и IP-телефонии.

В общем все версии протокола действуют по следующему алгоритму:

\begin{enumerate}
	\item Клиент подключается к серверу и отправляет ему поддерживаемую версию протокола TLS и поддерживаемые алгоритмы шифрования.
	\item Сервер отвечает, какую версию и какое шифрование он готов использовать.
	\item Сервер отправляет сертификат своего открытого ключа, если разделение общего ключа будет происходить через асимметричную криптографию.
	\item Сервер отправляет свою часть Диффи-Хеллмана, если разделение общего ключа будет происходить через DH.
	\item Клиент, в зависимости от способа разделения ключа, отправляет либо свою часть DH, либо зашифрованный на открытом ключе сервера секрет.
	\item При необходимости клиент и сервер завершают выработку общего секрета, на основе которого будет осуществляться дальнейшее шифрование всего.
	\item Клиент сообщает, что дальнейшая связь будет зашифрована и в зашифрованном виде отправляет хеш и имитовставку всех предыдущих сообщений.
	\item Сервер всё проверяет и отправляет то же самое.
	\item Клиент всё проверяет.
	\item Если все проверки успешны, соединение считается установленным и дальше вся передача данных зашифровывается на основе разделённого секрета. 
\end{enumerate}

От версии к версии в протоколе меняются алгоритмы шифрования, цифровой подписи и выработки имитовставки, рекомендованные к использованию. 

В частности, TLS 1.3 (последняя на данный момент версия) содержит алгоритмы из ГОСТ.