\section{Методы выработки производных ключей, принципы оценки качества производной ключевой информации.}

В России выработка производной ключевой информации регламентируется Рекомендациями по стандартизации Р 1323565.1.022--2018.

При реализации средствами криптографической защиты информации (СКЗИ) нескольких криптографических функций возникает необходимость использования для механизмов, реализующих каждую из функций, различных ключей, выработанных из исходной ключевой информации. Исходной ключевой информацией может являться, например, заранее распределенный ключ или ключ, полученный в результате выполнения протоколов выработки общего ключа.

Функции выработки производных ключей осуществляют криптографическое преобразование исходной ключевой информации с использованием дополнительных открытых данных с целью получения
ключей для дальнейшего использования в различных функциях.

В описании используются следующие обозначения:

\begin{itemize}
	\item $S$ -- исходная ключевая информация;
	\item $C$ -- представление числа, используемого в итеративных конструкциях в качестве счетчика. Способ представления счетчика должен быть согласован между участниками информационного обмена;
	\item $P$ -- метка использования -- двоичная строка, содержащая информацию об использовании вырабатываемых производных ключей. Может содержать, например, информацию о конкретном механизме, для которого предназначается производный ключ (ключ шифрования ключей, ключ шифрования данных, ключ имитозащиты и т. п.) или, в случае одновременной выработки ключей для нескольких примитивов, информацию о разделении производного ключевого материала на различные производные ключи; допустимые значения и способ представления должны быть согласованы между участниками информационного обмена;
	\item $U$ -- информация об участниках информационного обмена, которыми предполагается ис пользование вырабатываемой ключевой информации;
	\item может включать в себя идентификаторы пользователей и прочую информацию, известную всем участникам, вырабатывающим производную ключевую информацию;
	\item $A$ -- некоторая дополнительная информация, используемая при выработке производной ключевой информации, например, метка времени;
	\item $L$ -- длина (в двоичной записи) вырабатываемого производного ключевого материала в битах;
	\item $T$ -- соль -- случайная строка фиксированной длины, обычно вырабатываемая в момент выполнения алгоритма.
\end{itemize}

Функции выработки производных ключей принимают на вход шесть аргументов: исходную ключевую информацию $S$, длину производной ключевой информации $L$, соль $T$, метку использования $T$, информацию о субъектах $U$, дополнительную информацию $A$.

Функции состоят из двух этапов. На первом этапе из исходной ключевой информации и соли вырабатывается промежуточный ключ длины 256 бит. Полученный промежуточный ключ вместе с остальными входными параметрами используется на втором этапе функций, выходом которых является производный ключевой материал $K_0$ длины $L$. Первый этап функции обозначается $kdf^{(1)}$ и возвращает промежуточный ключ $K^{(1)}$, второй обозначается $kdf^{(2)}$ и возвращает производный ключ $K_0$.

В качестве промежуточных преобразований используют ключевые функции хеширования $HMAC_K^{(n)}$ и $NMAC_K^{(256)}$, вырабатывающие имитовставку длины $n$ и 256 бит соответственно. Обе эти функции в качестве базовой хеш-функции ($Hash$) используют хеш-функцию из ГОСТ Р 34.11-2012:

$$ HMAC_K^{(n)}(X) = Hash^{(n)} \left( (K \oplus C_{OUT}) || Hash^{(n)}((K \oplus C_{IN}) || X) \right) $$

$$ NMAC_K^{(256)}(X) = Hash^{(256)} \left( (K \oplus C_{OUT}) || Hash^{(512)}((K \oplus C_{IN}) || X) \right) $$

Здесь $C_{IN}$ и $C_{OUT}$ -- константы, заданные в стандарте, $||$ -- операция конкатенации битовых строк. 

В качестве функции выработки промежуточного ключа используется одна из следующих:

\begin{enumerate}
	\item $K^{(1)} = NMAC_T^{(256)}(S),\ \text{при этом}\ |T| \leq 512$
	\item $K^{(1)} = LB_{256} (HMAC_T^{(512)}(S)),\ \text{при этом}\ |T| \leq 512$
	\item $K^{(1)} = S \oplus T,\ \text{при этом}\ |T| = 256$
\end{enumerate}

Здесь $LB_{x}(X)$ -- взятие $x$ старших (левых) бит строки $X$, $|X|$ -- длина строки $X$.

Далее для выработки результирующего производного ключа используют следующий алгоритм:

\renewcommand{\labelenumii}{\arabic{enumi}.\arabic{enumii})}
\begin{enumerate}
	\item $C_0 = 0$
	\item $z_0 = IV$
	\item Для $i = \overline{1, \lceil L / n \rceil}$ выполнять:
	\begin{enumerate}
		\item $C_i = C_{i - 1} + 1$
		\item $z_i = MAC_{K^{(1)}}^{(n)}(format(z_i - 1, C_i, P, U, A, L))$
	\end{enumerate}
	\item $K_0 = LB_L \left( z_1 || z_2 || ... || Z_{\lceil L/n \rceil} \right)$		
\end{enumerate}

Здесь:

\begin{itemize}
	\item $MAC$ -- ключевая функция хеширования, в качестве неё можно взять $HMAC$, $NMAC$ или режим выработки имитовставки из ГОСТ Р 34.13-2015 (см. вопрос 16)
	\item $IV$ -- вектор инициализации -- общедоступное значение, которое открыто передаётся перед стартом протокола
	\item $format$ -- общеизвестный способ превратить 6 значений в строку бит (оговаривается сторонами заранее)
\end{itemize}

Основное требование к производной ключевой информации, применяемой для шифрования и имитозащиты передаваемых сообщений, заключается в невозможности её определения нарушителем с трудоёмкостью, меньшей чем тотальное опробование всех возможных значений. Каждый ключ, как правило, представляется в виде двоичного вектора длины $m$ бит, таким образом, нарушителю необходимо опробовать $2^m$ ключей для компрометации сообщений. Отсюда следует, что необходимо проверить выполнимость следующих условий:

\begin{itemize}
	\item Множество значений, которые может принимать производный ключ, совпадает с множеством $V_m$ всех возможных двоичных векторов длины $m$;
	\item Принимаемые производным ключом значения непредсказуемы, т. е. последовательность нескольких выработанных в различных сессиях протокола производных ключей $K1, K2, ...$ должна быть статистически неотличима от последовательности случайных равновероятно распределённых на множестве $V_m$ величин.
\end{itemize}

При практическом применении средств защиты информации могут нарушаться правила эксплуатации средств, превышаться заданные ограничения на объём обрабатываемой информации или возникать уязвимости в программном обеспечении, все вместе или по отдельности приводящие к возможности практического определения нарушителем производных ключей или исходной ключевой информации. Это приводит к необходимости встраивания в криптографические протоколы мер, минимизирующих объём скомпрометированной информации. В качестве таких мер могут выступать:

\begin{enumerate}
	\item использование односторонних функций, не позволяющих вычислять значения ключей аутентификации по значениям производных ключей; 
	\item использование в каждой сессии протокола уникальных случайных значений для выработки производных ключей; 
	\item использование <<древовидных>> структур выработки производных ключей, не позволяющих нарушителю по известному производному ключу $K_n$ вычислить значения ключей $K_{n-1}$ и $K_{n+1}$.
\end{enumerate}

Дополнительно в рамках математических исследований должны быть проверены следующие гипотезы:

\begin{itemize}
	\item о ничтожной вероятности совпадения различных производных ключей, вырабатываемых в рамках одной сессии протокола;
	\item о статистической независимости последовательности производных ключей $K_1, K_2, ...$, вырабатываемых в различных сессиях протокола.
\end{itemize}