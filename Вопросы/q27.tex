\section{Протоколы выработки общего ключа.}

Выработка общего ключа -- ситуация, когда до конца работы протокола ни у одной из сторон заранее нет ключа, и он появляется только как результат работы протокола. Не путать с распределением ключей -- ситуацией, когда ключ генерируется одной из сторон и задачей является <<раздать>> этот ключ всем остальным. 

Классика в выработке общего ключа, которая до сих пор используется с некоторыми изменениями -- протокол Диффи-Хеллмана (DH). Протокол предельно прост и включает следующие шаги:

\begin{enumerate}
	\item $A: \text{выбирает числа}\ g, p, \text{а также генерирует случайное число}\ a$
	\item $B: \text{генерирует случайное число}\ b$
	\item $A: \text{вычисляет}\ X = g^a\ mod\ p$
	\item $A \rightarrow B: g, p, X$
	\item $B: \text{вычисляет}\ Y = g^b\ mod\ p$
	\item $B \rightarrow A: Y$
	\item $A: \text{вычисляет}\ K = Y^a\ mod\ p = g^{ab}\ mod\ p$
	\item $B: \text{вычисляет}\ K = X^b\ mod\ p = g^{ab}\ mod\ p$
	\item Значение $K$, выработанное обеими сторонами, является общим секретным ключом
\end{enumerate}

Здесь:

\begin{itemize}
	\item $A$ и $B$ -- Алиса и Боб -- стороны информационного обмена.
	\item $p$ -- случайное простое число такое, что $\frac{p - 1}{2}$ -- тоже простое число.
	\item $g$ -- первообразный корень по модулю $p$ (тоже простое число).
	\item $a$ и $b$ -- большие случайные числа. 
\end{itemize}

В протоколе важно то, что абсолютно все передаваемые значения могут быть доступны злоумышленнику, и это никак не повлияет на безопасность. Однако это работает только в том случае, когда злоумышленник пассивен, т.е. может только прослушивать канал, но не перехватывать и не отправлять сообщения в нём. Т.е. протокол уязвим к атаке <<человек посередине>>. 

Чтобы избежать этой уязвимости вводятся модифицированные протоколы. Первый из них -- STS -- выглядит следующим образом (все предварительные вычисления на сторонах $A$ и $B$ остаются такими же):

\begin{enumerate}
	\item $A \rightarrow B: g^a\ mod\ p$
	\item $B \rightarrow A: g^b\ mod\ p, E_K(S_B(g^b, g^a))$
	\item $A \rightarrow B: E_K(S_A(g^a, g^b))$
\end{enumerate}

Здесь:

\begin{itemize}
	\item $S_A$ и $S_B$ -- цифровые подписи сторон $A$ и $B$.
	\item $E_K(x)$ -- зашифрование сообщения $x$ симметричным алгоритмом с использованием выработанного ключа $K$.  
\end{itemize}

Также следует отметить, что ключ $K$ становится известен стороне $B$ уже на шаге 2, стороне $A$ -- на шаге 3. Следовательно, на шаге 2 Боб уже может зашифровать сообщение на этом ключе, а Алиса, в свою очередь, может расшифровать его на шаге 3.

Цифровые подписи здесь используются для того, чтобы подтвердить, что соответствующие значения пришли именно от той стороны, от которой ожидалось, а не от <<человека посередине>>. Зашифрование используется, чтобы подтвердить, что значение ключа выработано верно.

Также есть модификация под названием MTI. В ней используется вариация криптографии с открытым ключом:

\begin{enumerate}
	\item $A: \text{выбирает секретный ключ}\ 1 \leq a \leq p - 2$
	\item $A: \text{публикует открытый ключ}\ Z_A = g^a\ mod\ p$
	\item $B: \text{выбирает секретный ключ}\ 1 \leq b \leq p - 2$
	\item $B: \text{публикует открытый ключ}\ Z_B = g^b\ mod\ p$
	\item $A: \text{генерирует случайное число}\ 1 \leq x \leq p - 2$
	\item $B: \text{генерирует случайное число}\ 1 \leq y \leq p - 2$
	\item $A \rightarrow B: g^x\ mod\ p$
	\item $B \rightarrow A: g^y\ mod\ p$
	\item Обе стороны вычисляют $K = (g^y)^aZ_B^x\ mod\ p = (g^x)^bZ_A^y\ mod\ p = g^{xb + ya}\ mod\ p$	
\end{enumerate}

Публикация открытых ключей приводит к тому, что любая подмена приводит к неверным ключам, и никто ничего не сможет расшифровать.

Также существует модификация DH на эллиптических кривых. В ней всё то же самое за исключением того, что вместо возведения числа в степень по модулю используется умножение точки эллиптической кривой на число. 



